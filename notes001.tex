%
% 底本: 史記         from 點校本  中華書局
%       史記會注考證 from 影印本  文學古籍出版社
%       史記會注考證 from 楊海崢  上海古籍出版社(改正本)
% 錄入: 史記三家注   from 電子版  中國哲學書電子化計劃
%
% 錄入確認: 1
% 用字確認: 1
% 句讀確認: 1
% 專名確認: 1
% 注釋完成: 1
%
% ---------- Start ----------

%<*InsertedNotes:001:01>
\KAOZ 司馬貞唐人,諱「世」作「系」,下文系本亦世本也,全文倣之。
%     ¯¯¯¯¯¯ ¯                          ˜˜˜˜  ˜˜˜˜
%</InsertedNotes:001:01>

%<*InsertedNotes:001:02>
\KAOZ 事見禮樂記、史記樂書。
%         ˜˜ ˜˜˜  ˜˜˜˜ ˜˜˜
%</InsertedNotes:001:02>

%<*InsertedNotes:001:03>
\KAOZ 乾隆四年經史舘校刊本「音」作「旨」。
%     ¯¯¯¯
%</InsertedNotes:001:03>

%<*InsertedNotes:001:04>
\KAOZ 錢大昕曰:司馬貞注高祖紀「母曰劉媼」云:「今近有人云『母溫氏』。貞時打得
%     ¯¯¯¯¯¯    ¯¯¯¯¯¯  ˜˜˜˜˜˜      ¯¯¯¯                      ¯¯¯¯    ¯¯
班固泗水亭長古碑文,其字分明作『溫』字,云『母溫氏』。貞與賈膺復、徐彥伯、魏奉
%¯¯¯ ¯¯¯¯¯                                    ¯¯¯¯    ¯¯  ¯¯¯¯¯¯  ¯¯¯¯¯¯  ¯¯¯¯
古等執對反覆,沈歎古人未聞。」按:「膺復」當作「膺福」。先天二年,為右散騎常侍、
%¯                                                      ¯¯¯¯
昭文舘學士,以預太平公主逆謀誅。今河内縣有大雲寺碑,即膺福書也。徐彥伯卒與開元
%¯¯¯¯¯          ¯¯¯¯¯¯¯¯          ¯¯¯¯¯¯  ˜˜˜˜˜˜˜˜    ¯¯¯¯      ¯¯¯¯¯¯    ¯¯¯¯
二年,見唐書本傳。而司馬貞、張守節二人,新、舊唐書無傳,守節正義序稱「開元二十
%       ˜˜˜˜        ¯¯¯¯¯¯  ¯¯¯¯¯¯      ˜˜  ˜˜˜˜˜˜      ¯¯¯¯ ˜˜˜˜˜    ¯¯¯¯
四年八月,殺青斯竟」,而小司馬兩序則不載譔述年月,以此注驗之,其與賈、徐諸人談
%                       ¯¯¯¯¯¯                                    ¯¯  ¯¯
議,當在中、睿之世,計其年輩,似在張守節之前。補史記序自題「國子博士、弘文舘學
%       ¯¯  ¯¯                    ¯¯¯¯¯¯      ˜˜˜˜˜˜˜˜                ¯¯¯¯¯¯
士」。唐制,弘文舘皆以他官兼領,五品以上為學士,六品以下曰直學士。國子博士,係
%     ¯¯    ¯¯¯¯¯¯
正五品上,故得學士之稱。神龍以後,避孝敬皇帝諱,或稱昭文,或稱修文。開元七年,
%                       ¯¯¯¯        ¯¯¯¯¯¯¯¯        ¯¯¯¯      ¯¯¯¯  ¯¯¯¯
仍為弘文。小司馬充學士,蓋在開元七年以後也。唐書劉知幾傳:「開元初嘗議孝經鄭氏
%   ¯¯¯¯  ¯¯¯¯¯¯            ¯¯¯¯            ˜˜˜˜ ˜˜˜˜˜˜˜    ¯¯¯¯      ˜˜˜˜ ¯¯¯
學,非康成注,當以古文為正。易無子夏傳,老子書無河上公注,請存王弼學。宰相宋璟
%     ¯¯¯¯                  ˜˜  ¯¯¯¯    ˜˜˜˜    ¯¯¯¯¯¯        ¯¯¯¯        ¯¯¯¯
等不然其論,奏與諸儒辨質,博士司馬貞等共黜其言,請二家兼行,唯子夏易傳請罷。詔
%                             ¯¯¯¯¯¯                          ¯¯¯¯ ˜˜˜
可。」又考唐藝文志稱貞「開元潤州別駕」,蓋由弘文舘出為別駕,遂蹭蹬以死也。
%         ¯¯ ˜˜˜˜˜  ¯¯  ¯¯¯¯ ¯¯¯            ¯¯¯¯¯¯
%</InsertedNotes:001:04>

% ---------- End ----------
