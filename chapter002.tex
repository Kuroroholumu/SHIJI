%
% 底本: 史記         from 點校本  中華書局
%       史記會注考證 from 影印本  文學古籍出版社
%       史記會注考證 from 楊海崢  上海古籍出版社(改正本)
% 錄入: 史記三家注   from 電子版  中國哲學書電子化計劃
%
% 錄入確認: 1
% 用字確認: 1
% 句讀確認: 1
% 專名確認: 1
% 注釋完成: 1
%
% ---------- Start ----------

% 史記索隱後序

夫太史公紀事,上始軒轅,下訖天漢,雖博采古文及傳記諸子,其閒殘闕蓋多,或旁搜異
% ¯¯¯¯¯¯          ¯¯¯¯      ¯¯¯¯
聞以成其說,然其人好奇而詞省,故事覈而文微,是以後之學者多所未究。其班氏之書,
%                                                                   ¯¯¯¯
成於後漢。彪旣後遷而述,所以條流更明,且又兼采衆賢,羣理畢備,
%   ¯¯¯¯  ¯¯    ¯¯
\loadnotes{../contents/notes002}{InsertedNotes:002:01}%
%
故其旨富,其詞文,是以近代諸儒共所鑽仰。
%
\loadnotes{../contents/notes002}{InsertedNotes:002:02}%
%
其訓詁蓋亦多門,蔡謨集解之時已有二十四家之說,所以於文無所滯,於理無所遺。而太
%               ¯¯¯¯                                                        ¯¯
史公之書,既上序軒黃,中述戰國,或得之於名山壞宅,或取之以舊俗風謠,
%¯¯¯            ¯¯¯¯      ¯¯¯¯
\loadnotes{../contents/notes002}{InsertedNotes:002:03}%
%
故其殘文斷句難究詳矣。
%
\parswitch
%
然古今為注解者絕省,音義亦希。始後漢延篤乃有音義一卷,又別有音隱五卷,不記作者
%                               ¯¯¯¯ ¯¯¯    ˜˜˜˜            ˜˜˜˜
何人,
%
\loadnotes{../contents/notes002}{InsertedNotes:002:04}%
%
近代鮮有二家之本。宋中散大夫徐廣作音義十三卷,
%                 ¯¯        ¯¯¯¯  ˜˜˜˜
\loadnotes{../contents/notes002}{InsertedNotes:002:05}%
%
唯記諸家本異同,於義少有解釋。又中兵郎裴駰,亦名家之子也,作集解注本,合為八十
%                                     ¯¯¯¯                  ˜˜˜˜
卷,見行於代。仍云亦有音義,前代久已散亡。南齊輕車錄事鄒誕生亦撰音義三卷,音則
%                     ˜˜˜˜                ¯¯¯¯        ¯¯¯¯¯¯    ˜˜˜˜
尚奇,義則罕說。
%
\loadnotes{../contents/notes002}{InsertedNotes:002:06}%
%
隋祕書監柳顧言尤善此史。劉伯莊云,其先人曾從彼公受業,或音解隨而記錄,凡三十卷。
%¯      ¯¯¯¯¯¯          ¯¯¯¯¯¯
隋季喪亂,遂失此書。伯莊以貞觀之初,奉勑於弘文館講授,遂采鄒.徐二說,兼記憶柳
%¯                  ¯¯¯¯  ¯¯¯¯            ¯¯¯¯¯¯          ¯¯  ¯¯            ¯¯
公音旨,遂作音義二十卷。音乃周備,義則更略。惜哉!
%¯          ˜˜˜˜
\loadnotes{../contents/notes002}{InsertedNotes:002:07}%
%
古史微文,遂由數賢祕寶,故其學殆絕。
%
\parswitch
%
前朝吏部侍郎許子儒亦作注義,不覩其書。崇文館學士張嘉會獨善此書,而無注義。貞少
%           ¯¯¯¯¯¯    ˜˜˜˜            ¯¯¯¯¯¯    ¯¯¯¯¯¯                    ¯¯
從張學,晚更研尋,初以殘闕處多,兼鄙褚少孫誣謬,因憤發而補史記,遂兼注之,然其
% ¯¯                                ¯¯¯¯¯¯                ˜˜˜˜
功殆半。乃自惟曰:「千載古史,良難紬繹。」於是更撰音義,重作述贊,蓋欲以剖盤根
%                                                 ˜˜˜˜
之錯節,遵北轅於司南也。凡為三十卷,號曰史記索隱云。
%                                       ˜˜˜˜˜˜˜˜
\loadnotes{../contents/notes002}{InsertedNotes:002:08}%

% ---------- End ----------
