%
% 底本: 史記         from 點校本  中華書局
%       史記會注考證 from 影印本  文學古籍出版社
%       史記會注考證 from 楊海崢  上海古籍出版社(改正本)
% 錄入: 史記三家注   from 電子版  中國哲學書電子化計劃
%
% 錄入確認: 1
% 用字確認: 1
% 句讀確認: 1
% 專名確認: 1
% 注釋完成: 1
%
% ---------- Start ----------

% 史記索隱序
% 朝散大夫國子博士弘文館學士河內司馬貞

史記者,漢太史司馬遷父子之所述也。遷自以承五百之運,繼春秋而纂是史,其襃貶覈實
%˜˜˜    ¯¯    ¯¯¯¯¯¯              ¯¯                  ˜˜˜˜
頗亞於丘明之書,於是上始軒轅,下訖天漢,作十二本紀、十表、八書、三十系家、七十
%     ¯¯¯¯              ¯¯¯¯      ¯¯¯¯
列傳,凡一百三十篇,
%
\loadnotes{../contents/notes001}{InsertedNotes:001:01}%
%
始變左氏之體,而年載悠邈,簡冊闕遺,勒成一家,其勤至矣。又其屬稾先據左氏、國語、
%   ˜˜˜˜                                                            ˜˜˜˜  ˜˜˜˜
系本、戰國策、楚漢春秋及諸子百家之書,而後貫穿經傳,馳騁古今,錯綜櫽括,各使成
%˜˜˜  ˜˜˜˜˜˜  ˜˜˜˜˜˜˜˜
一國一家之事,故其意難究詳矣。比於班書,微為古質,故漢晉名賢未知見重,所以魏文
%                                 ¯¯ ˜              ¯¯ ¯                  ¯¯¯¯
侯聽古樂則唯恐臥,
%¯
\loadnotes{../contents/notes001}{InsertedNotes:001:02}%
%
良有以也。
%
\parswitch
%
逮至晉末,有中散大夫東莞徐廣始考異同,作音義十三卷。宋外兵參軍裴駰又取經傳訓釋
%   ¯¯              ¯¯¯¯ ¯¯¯            ˜˜˜˜        ¯¯        ¯¯¯¯
作集解,合為八十卷。雖麤見微意,而未窮討論。南齊輕車錄事鄒誕生亦作音義三卷,音
% ˜˜˜˜                                      ¯¯¯¯        ¯¯¯¯¯¯    ˜˜˜˜
則微殊,義乃更略。爾後其學中廢。貞觀中,諫議大夫崇賢館學士劉伯莊達學宏才,鉤深
%                               ¯¯¯¯            ¯¯¯¯¯¯    ¯¯¯¯¯¯
探賾,又作音義二十卷,比於徐、鄒,音則具矣。殘文錯節,異音微義,雖知獨善,不見
%         ˜˜˜˜            ¯¯  ¯¯
旁通,
%
\loadnotes{../contents/notes001}{InsertedNotes:001:03}%
%
欲使後人從何準的。
%
\parswitch
%
貞謏聞陋識,頗事鑽研。而家傳是書,不敢失墜。初欲改更舛錯,裨補疏遺,義有未通,
%¯
兼重注述。然以此書殘缺雖多,實為古史,忽加穿鑿,難允物情。今止探求異聞,採摭典
%
故,解其所未解,申其所未申者,釋文演注,又重為述贊,凡三十卷,號曰史記索隱。雖
%                                                                 ˜˜˜˜˜˜˜˜
未敢藏之書府,亦欲以貽厥孫謀云。
%
\loadnotes{../contents/notes001}{InsertedNotes:001:04}%

% ---------- End ----------
