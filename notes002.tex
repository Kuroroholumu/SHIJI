%
% 底本: 史記         from 點校本  中華書局
%       史記會注考證 from 影印本  文學古籍出版社
%       史記會注考證 from 楊海崢  上海古籍出版社(改正本)
% 錄入: 史記三家注   from 電子版  中國哲學書電子化計劃
%
% 錄入確認: 1
% 用字確認: 1
% 句讀確認: 1
% 專名確認: 1
% 注釋完成: 1
%
% ---------- Start ----------

%<*InsertedNotes:002:01>
\KAOZ 索隱單本「且由」作「是」。
%     ˜˜˜˜
%</InsertedNotes:002:01>

%<*InsertedNotes:002:02>
\KAOZ 單本「所」作「行」。
%
%</InsertedNotes:002:02>

%<*InsertedNotes:002:03>
\KAOZ 單本「宅」作「壁」。
%
%</InsertedNotes:002:03>

%<*InsertedNotes:002:04>
\KAOZ 單本「音隱」作「章隱」。
%
%</InsertedNotes:002:04>

%<*InsertedNotes:002:05>
\KAOZ 張文虎曰:十三卷,原誤「一十卷」,依前序及集解序正義改,唐志亦云十三卷。
%     ¯¯¯¯¯¯                                    ˜˜˜˜˜˜ ˜˜˜    ˜˜˜˜
%</InsertedNotes:002:05>

%<*InsertedNotes:002:06>
\KAOZ 隋經籍志、日本現在書目作「梁輕車錄事參軍鄒誕生」。釋玄應一切經音義引誕生
%     ¯¯ ˜˜˜˜˜  ˜˜˜˜˜˜˜˜˜˜˜˜    ¯¯            ¯¯¯¯¯¯    ¯¯¯¯¯¯ ˜˜˜˜˜˜˜˜˜  ¯¯¯¯
史記音。
%˜˜˜˜˜
%</InsertedNotes:002:06>

%<*InsertedNotes:002:07>
\KAOZ 日本現在書目云:「史記音義廿卷,唐大中大夫劉伯莊撰。」二十卷,諸本作「三
%     ˜˜˜˜˜˜˜˜˜˜˜˜      ˜˜˜˜˜˜˜˜      ¯¯        ¯¯¯¯¯¯
十卷」,今從金陵書局本。
%           ¯¯¯¯¯¯¯¯
%</InsertedNotes:002:07>

%<*InsertedNotes:002:08>
\KAOZ 單本「惟」作「唯」,「紬繹」作「閒然」,「於是更」作「因退」,「述贊」作
%
「贊述」。
%
%</InsertedNotes:002:08>
